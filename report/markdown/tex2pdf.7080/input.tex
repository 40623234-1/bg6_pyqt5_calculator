\documentclass[12pt,,]{report}
\usepackage{lmodern}
\usepackage{amssymb,amsmath}
\usepackage{ifxetex,ifluatex}
\usepackage{fixltx2e} % provides \textsubscript
\ifnum 0\ifxetex 1\fi\ifluatex 1\fi=0 % if pdftex
  \usepackage[T1]{fontenc}
  \usepackage[utf8]{inputenc}
\else % if luatex or xelatex
  \ifxetex
    \usepackage{mathspec}
  \else
    \usepackage{fontspec}
  \fi
  \defaultfontfeatures{Ligatures=TeX,Scale=MatchLowercase}

    \usepackage{xeCJK}
    % 中文自動換行
    \XeTeXlinebreaklocale "zh"
    % 文字的彈性間距
    \XeTeXlinebreakskip = 0pt plus 1pt
    \newfontlanguage{Chinese}{CHN}
    % 章次20級,節次16級,小節次以下14級,本文12級字
    \def\LARGE{\fontsize{20}{30}\selectfont}%章次
    \def\Large{\fontsize{16}{24}\selectfont}%節次
    \def\large{\fontsize{14}{21}\selectfont}%小節次
    \usepackage{indentfirst}
    \usepackage{CJKnumb}
    \renewcommand{\figurename}{圖}
    \renewcommand{\thefigure}{{\arabic{chapter}}.\arabic{figure}}
    \renewcommand{\tablename}{表}
    \renewcommand{\thetable}{{\arabic{chapter}}.\arabic{table}}
    %重製章節
    \renewcommand{\chaptername}{}
    \renewcommand{\thechapter}{第\CJKnumber{\arabic{chapter}}章}
    \renewcommand{\thesection}{{\arabic{chapter}}.\arabic{section}}
    \renewcommand{\thesubsection}{{\arabic{chapter}}.{\arabic{section}}.\arabic{subsection}}
    %設定行距與中英文字型
    \linespread{1}\selectfont
    \setCJKmainfont{SimSun}
    \setmainfont{Times New Roman}
    \setromanfont{Times New Roman}
    \setmonofont{Times New Roman}
    %重製章節標籤
    \usepackage{titlesec}
    \titleformat{\chapter}[block]{\LARGE\centering}{\thechapter}{0.5em}{}
    \titleformat{\section}[block]{\Large}{\thesection}{0.5em}{}
    \titleformat{\subsection}[block]{\large}{\thesubsection}{0.5em}{}
    % 重製目錄
    \usepackage{titletoc}
    \titlespacing{\chapter}{0pt}{*0}{*2}
    \titlespacing{\section}{0pt}{*1}{*1}
    \titlespacing{\subsection}{0pt}{*1}{*1}
    \titlespacing{\subsubsection}{0pt}{*1}{*1}
    \titlecontents{chapter}[0em]{}{\contentspush{\thecontentslabel}\hspace*{1em}}{}{\titlerule*[0.7pc]{.}\contentspage}
\fi
% use upquote if available, for straight quotes in verbatim environments
\IfFileExists{upquote.sty}{\usepackage{upquote}}{}
% use microtype if available
\IfFileExists{microtype.sty}{
\usepackage{microtype}
\UseMicrotypeSet[protrusion]{basicmath} % disable protrusion for tt fonts
}{}
\usepackage[margin=1in]{geometry}
\usepackage[unicode=true]{hyperref}
\hypersetup{
            pdfauthor={設計一乙 40623225 卓昆峰; 設計一乙 40623226 鄭清詮; 設計一乙 40623227 張耀元; 設計一乙 40623234 洪一木; 設計一乙 40623235 黃昱誠; 設計一乙 40623236 黃子峰},
            pdfborder={0 0 0},
            breaklinks=true}
\urlstyle{same}  % don't use monospace font for urls
\ifnum 0\ifxetex 1\fi\ifluatex 1\fi=0 % if pdftex
  \usepackage[shorthands=off,main=]{babel}
\else
  \usepackage{polyglossia}
  \setmainlanguage[]{}
\fi
\IfFileExists{parskip.sty}{%
\usepackage{parskip}
}{% else
\setlength{\parindent}{0pt}
\setlength{\parskip}{6pt plus 2pt minus 1pt}
}
\setlength{\emergencystretch}{3em}  % prevent overfull lines
\providecommand{\tightlist}{%
  \setlength{\itemsep}{0pt}\setlength{\parskip}{0pt}}
\setcounter{secnumdepth}{5}
% Redefines (sub)paragraphs to behave more like sections
\ifx\paragraph\undefined\else
\let\oldparagraph\paragraph
\renewcommand{\paragraph}[1]{\oldparagraph{#1}\mbox{}}
\fi
\ifx\subparagraph\undefined\else
\let\oldsubparagraph\subparagraph
\renewcommand{\subparagraph}[1]{\oldsubparagraph{#1}\mbox{}}
\fi

% set default figure placement to htbp
\makeatletter
\def\fps@figure{htbp}
\makeatother


\begin{document}
%Cover Start
\begin{titlepage}
\vspace{1cm}
\begin{center}
\fontsize{36}{54}\selectfont{
    國立虎尾科技大學\par
}
\fontsize{28}{42}\selectfont{機械設計工程系\par}
\fontsize{24}{36}\selectfont{計算機程式 bg6 期末報告\par}
\vspace{1.5cm}
\fontsize{20}{30}\selectfont{
    PyQt5 事件導向計算器\par
    PyQt5 Event-Driven Calculator Project\par
}
\vspace{\fill}
\fontsize{18}{27}\selectfont{
    學生:\par
    設計一乙 40623225 卓昆峰 \par 設計一乙 40623226 鄭清詮 \par 設計一乙 40623227 張耀元 \par 設計一乙 40623234 洪一木 \par 設計一乙 40623235 黃昱誠 \par 設計一乙 40623236 黃子峰 \par
    指導教授:嚴家銘\par
}
\vspace{1.5cm}
\fontsize{16}{24}\selectfont{2017.12.18\par}
\end{center}
\vspace{1cm}
\end{titlepage}

\newcommand\frontmatter{
    \cleardoublepage
    \pagenumbering{roman}
}

\newcommand\mainmatter{
    \cleardoublepage
    \pagenumbering{arabic}
}

\newcommand\backmatter{
    \if@openright
        \cleardoublepage
    \else
        \clearpage
    \fi
}

%Document start

% Set page number to arabic i ii...
\frontmatter
\renewcommand{\abstractname}{\LARGE \center 摘要}
\chapter*{摘要}
\addcontentsline{toc}{chapter}{摘要}
\fontsize{14}{21}\selectfont{本研究的重點在於每組寫一個計算機程式,並分配每組每個組員需要負責的按鍵,然後依照老師所給的範例寫入各按鍵的程式,這並非只是抄襲,而是在寫入每行程式之後,並且去理解每行的程式運作及方法,然後將遇到的問題組別一起討論,並將所遇到的問題解決,讓自己本身以及自己組員對計算機程式有更深的了解。

\begin{itemize}
\item
  我們使用eric6來做計算機程式
\item
  使用github來做協同
\end{itemize}}


\begingroup
    \renewcommand{\contentsname}{\center 目錄 \addcontentsline{toc}{chapter}{目錄}}
    \renewcommand{\numberline}[1]{~#1\hspace*{1em}}
        \setcounter{tocdepth}{2}
    \tableofcontents
    \newcommand{\lotlabel}{表}
    \renewcommand{\listtablename}{\center 表目錄 \addcontentsline{toc}{chapter}{表目錄}}
    \renewcommand{\numberline}[1]{\lotlabel~#1\hspace*{1em}}
    \listoftables
    \newcommand{\loflabel}{圖}
    \renewcommand{\listfigurename}{\center 圖目錄 \addcontentsline{toc}{chapter}{圖目錄}}
    \renewcommand{\numberline}[1]{\loflabel~#1\hspace*{1em}}
    \listoffigures
\endgroup

% Start normal page number, 1 2 3
\mainmatter
\hypertarget{ux524dux8a00}{%
\chapter{前言}\label{ux524dux8a00}}

前言

為了讓我們知道以後協同的慨念和整體運作的方法所以使用MIKTex做報告
把我們一學期學到的做出整合分配協同出一個可用的計算機

機械設計就是靈活運用六種表達,
明確說明如何透過固體、流體與軟體元件之互動運作,
而能達成預定結果之明確與具體表達.

設計是一種明確與具體的表達, 而且是在仔細思考、多方考量後所完成的表達,
表達具有六種形式, 包括口語、文字、2D、3D、數學與實體表達,
設計的結果可以讓執行者有所依循, 根據指示執行後, 可得預期之結果.

機械是一種器物, 而且是由固體、流體與軟體元件精巧組合而成, 可互動運作,
達成特定功能之器物

三種類型的電腦與相關交互輔助設計模式:

近端(local) - 工程師面前的電腦, 以及機械設計系內部網路上連線的電腦.

遠端(cloud) - 廣域網路上的電腦.

可攜系統(mobile) - 工程師可以隨身啟動, 在各近端與遠端間移動,
仍能保有客製化的設定環境.

\hypertarget{ux53efux651cux7a0bux5f0fux7cfbux7d71ux4ecbux7d39}{%
\chapter{可攜程式系統介紹}\label{ux53efux651cux7a0bux5f0fux7cfbux7d71ux4ecbux7d39}}

可攜程式系統介紹

\hypertarget{ux555fux52d5ux8207ux95dcux9589}{%
\section{啟動與關閉}\label{ux555fux52d5ux8207ux95dcux9589}}

start.bat 會開啟cover\_and\_abstract cmd exe

可以從cover\_and\_abstract 中修改 start.bat讓他開啟時可以一起開啟leo

從cover\_and\_abstract 中修改 leo 讓他在 start.bat
開啟時可以找到要開的檔案

stop.bat 的作用關閉全部的檔案

\hypertarget{ux53efux651cux5f0fux7cfbux7d71ux4ecbux7d39}{%
\section{可攜式系統介紹}\label{ux53efux651cux5f0fux7cfbux7d71ux4ecbux7d39}}

可攜式系統的意義:為了能讓自己隨身隨地完成自己的工作
和建立自己習慣的開發方式

miktex:讓文字檔跟圖檔轉成PDF檔

GIMP2:修剪圖片裁切圖片

DiaPortable:可繪製圖形註解圖片

github:組態管理的一種 能讓多人協同 優點公開不用錢

Python36:在不同電腦能進行Python程式開發

Share-X:錄製影片 截圖

\hypertarget{calculator-ux7a0bux5f0f}{%
\chapter{Calculator 程式}\label{calculator-ux7a0bux5f0f}}

Calculator 程式細部說明

\hypertarget{ux8a08ux7b97ux6a5fux7a0bux5f0fux5167ux5bb9}{%
\section{計算機程式內容}\label{ux8a08ux7b97ux6a5fux7a0bux5f0fux5167ux5bb9}}

for i in num\_button: i.clicked.connect(self.digitClicked)

迴圈數字鍵 點案的時候連結到 digitClicked

self.display.setText(``0'')

起始 display 為 0

class Dialog(QDialog, Ui\_Dialog)

Dialog 類別同時繼承 QDialog 與 Ui\_Dialog 類別

self.equalButton.clicked.connect(self.equalClicked)

當你按到equalButton連結到equalClicked方法

self.waitingForOperand = True

起始時, 等待使用者輸入運算數值變數為真

def digitClicked(self)

使用者點擊按鈕時送出的按鈕指標類別, 在此利用此按鍵類別建立案例

clickedButton = self.sender()

clickedButton 即為當下使用者所按下的按鈕物件

if self.pendingAdditiveOperator: if not self.calculate(operand,
self.pendingAdditiveOperator):

\begin{verbatim}
            self.abortOperation()
            
            return
            
        self.display.setText(str(self.sumSoFar))
        
    else:

    self.pendingAdditiveOperator = clickedOperator
   
    self.waitingForOperand = True

        顯示目前的運算結果
        
        假如 self.pendingAdditiveOperator 為 False, 則將運算數與 self.fumSoFar 對應
        
        self.sumSoFar = operand
        
    能夠重複按下加或減, 以目前的運算數值執行重複運算
  
    進入等待另外一個運算數值的階段, 設為 True 才會清空 LineEdit
\end{verbatim}

if self.waitingForOperand: return

\begin{verbatim}
    self.display.setText('0')
    
    self.waitingForOperand = True
\end{verbatim}

在等待運算數階段, 直接跳出 slot, 不會清除顯示幕

清除顯示幕後, 重置等待運算數狀態變數

\hypertarget{python-ux7a0bux5f0fux8a9eux6cd5}{%
\chapter{Python 程式語法}\label{python-ux7a0bux5f0fux8a9eux6cd5}}

Python 程式語法

\hypertarget{ux8b8aux6578ux547dux540d}{%
\section{變數命名}\label{ux8b8aux6578ux547dux540d}}

Python3 變數命名規則與關鍵字 Python 英文變數命名規格

變數必須以英文字母大寫或小寫或底線開頭 變數其餘字元可以是英文大小寫字母,
數字或底線 變數區分英文大小寫 變數不限字元長度
不可使用關鍵字當作變數名稱

Python3 的程式關鍵字, 使用者命名變數時, 必須避開下列保留字.

Python keywords: {[}`False', `None', `True', `and', `as', `assert',
`break', `class', `continue', `def', `del', `elif', `else', `except',
`finally', `for', `from', `global', `if', `import', `in', `is',
`lambda', `nonlocal', `not', `or', `pass', `raise', `return', `try',
`while', `with', `yield'{]}

使用有意義且適當長度的變數名稱, 例如: 使用 length 代表長度,
不要單獨使用 l 或 L, 也不要使用 this\_is\_the\_length
程式前後變數命名方式盡量一致, 例如: 使用 rect\_length 或 RectLength
用底線開頭的變數通常具有特殊意義

\hypertarget{print-ux51fdux5f0f}{%
\section{print 函式}\label{print-ux51fdux5f0f}}

---def hi (): print(hi)

\hypertarget{ux91cdux8907ux8ff4ux5708}{%
\section{重複迴圈}\label{ux91cdux8907ux8ff4ux5708}}

---for i in num\_button: i.clicked.connect(self.digitClicked)

\hypertarget{ux5224ux65b7ux5f0f}{%
\section{判斷式}\label{ux5224ux65b7ux5f0f}}

---if self.pendingMultiplicativeOperator: if not self.calculate(operand,
self.pendingMultiplicativeOperator): self.abortOperation() return

\hypertarget{ux6578ux5217}{%
\section{數列}\label{ux6578ux5217}}

---num\_button = {[}self.one, self.two,\\
self.three, self.four, self.five, self.six, self.seven, self.eight,
self.nine, self.zero{]}

\hypertarget{pyqt5}{%
\chapter{PyQt5}\label{pyqt5}}

\hypertarget{pyqt5-ux67b6ux69cb}{%
\section{PyQt5 架構}\label{pyqt5-ux67b6ux69cb}}

精巧的電子零件,是透過輸入 010101,稱為「機器碼 (Machine
code)」的指令才能運作的。電子裝置可以直接辨識與執行機器碼的指令,而且不同的裝置使用的機器碼語言是不一樣的。

想當然爾,人類不可能一直查詢機器碼,再輸入給電子零件,因此才產生「程式
(Program)」。

組合語言 (Combination language)
是一種用於可編成組件的程式語言,一種組合語言專用於某種電腦的系統結構。換句話說,組合語言僅次於機器碼,可以以較簡單的方式命令,而且開發人員可以輕易辨別位址與數值

C 語言透過「編譯
(Compile)」這個流程,將程式碼轉為機器碼執行。不同作業系統存在相應的「編譯器
(Compiler)」,對應到硬體的機器碼執行,而某些編譯器正是組合語言撰寫的。

Python 語言是透過 C
語言編寫的直譯器進行轉譯與執行,因此不須經過編譯動作,甚至可以藉由伺服器在網路端執行,將結果回傳給客戶端。

低階語言(C)有以下特性:

處理範圍廣,而且可以直接影響運作效能。
較不易閱讀與開發,一件工作必須寫得較為詳細。

高階語言(Python)則反之:

處理範圍和運作效能端看基礎架構而定,大量運算時效能不如低階語言。
簡單且優化過的內建命令,往往會有簡單的表示法與程式庫、模組。

PyQt 幾乎支援 Qt 大部分的功能,並且將較專門的功能另外分成 PyQt Chart(2D
圖表)、PyQt Data Visualization(3D 圖表)、PyQt
Purchasing(應用程式購買功能)。

另外 QScintilla 是一個將 Scintilla 連結至 PyQt 的套件(在 C++ 可以直接用
Qt 和 Scintilla 即可),用途是辨識文字中的程式語言,以亮顯 (highlight)
的方式呈現,可以用作程式語言的辨識功能。

導入 sys 模組

import sys

導入 keyword 模組

import keyword

print() 函式用法

print() 為 Python 程式語言中用來列印數值或字串的函式, 其中有 sep
變數定義分隔符號, sep 內定為 ``,'', end 變數則用來定義列印結尾的符號,
end 內定為跳行符號.

簡單的 for 迴圈範例

for i in range(10): print(i)

函式用法與呼叫 使用者可以利用下列程式, 練習 def 函式定義與呼叫的用法.

定義函式

def square\_of\_x(x): return x*x

呼叫函式

y = square\_of\_x(3) print(y)

列印 y 對應內容

\hypertarget{ux5fc3ux5f97}{%
\chapter{心得}\label{ux5fc3ux5f97}}

期末報告心得

\hypertarget{fossil-scm}{%
\section{Fossil SCM}\label{fossil-scm}}

40623236:組成內容與狀態的配置, 因此軟體組態管理 就是針對軟體開發過程,
有關組成內容與狀態配置的管理

40623235:對於在配置上,有個方便的儲存與檢視的地方,不怕被刪檔,可靠而且對於程式設計師而言,日常工作中最常使用的工具,可能會是編輯器,或專為某種程式語言所設計的整合開發環境;而對負責大型軟體開發工作的軟體團隊成員來說,版本控制系統則是另一套相當重要的軟體工具。

40623234:一個相當簡單明瞭的程式,快速方便的就像一個大倉庫,隨時隨地的能將資料上傳與複製。

40623227:

40623226:

40623225:一個簡單可以依賴的一個軟體配置管理,這個是非常重要的

\hypertarget{ux7db2ux8a8cux5fc3ux5f97}{%
\section{網誌心得}\label{ux7db2ux8a8cux5fc3ux5f97}}

40623236:更新網誌是一件很麻煩的事 但是在打的過程中可以回復今天上的東西
以後旺季的時候也可以回去觀看複習 走過必留下痕跡
在網誌中遇到的問題難保以後不會遇到 這樣多去看已定可以更加進步

40623235:到期中考之前,我都還在摸索,可是在那之後,我知道他的重要性,而在裡面的內容,是關於以前學習的歷程,還有碰到的問題,都一一的在各個部分都清清楚楚的表示出來,慢慢的,會更加的充實

40623234:網誌能記錄一切所有的學習歷程,雖然在上傳時較不方便也較複雜,但整體上去相當的實用,尤其是最下面的討論區。

40623227:

40623226:

40623225:一開始學如何修改網誌的時候,非常讓人頭疼,可是越來越常修改的時候,就覺得已經開始熟悉它的步驟

\hypertarget{github-ux5354ux540cux5009ux5132}{%
\section{Github 協同倉儲}\label{github-ux5354ux540cux5009ux5132}}

bg6 協同倉儲:https://github.com/40623234-1/bg6\_pyqt5\_calculator

40623236:在推東西的時後要先記得pull 雖然協同很方便但是溝通是不可或缺的

40623235:Github協同倉儲有多彈性,還有做新增與修改來說,它是一個相當實用的軟體,而裡面開發的東西,是一些人共同要達成的目標

40623234:github是相當方便的倉儲,比fossil更進一步,能與他人進行同步,並用pull與push進行更新與上傳,方便許多。

40623227:

40623226:

40623225:協同倉儲可以讓在不同地方的人進行修改推送等功能,非常的方便快速

\hypertarget{ux5b78ux54e1ux5fc3ux5f97}{%
\section{學員心得}\label{ux5b78ux54e1ux5fc3ux5f97}}

40623236:協同是一個很重要的東西 一個東西完成需要很多人的幫忙和修改

40623235:和別人一起合作與探討,然而完成一件事情,是很有成就感的,
組員間互相支援,互相的討論,而且也有展現自我的空間,陪著組員一步一步慢慢做,
可以發現自己跟別人都可以慢慢的進步,雖然犯了錯,也是進步的關鍵

40623234:在一開始時,我剛進這間學校後發現有計算機程式這堂課,
那一瞬間想到高職時尚的計算機概論, 差不多是了解電腦程式和灌灌程式,
後來知道其實fossil與github的方便性與快速性,
了解為何世界的科技能進步如此快速。

40623227:因為是初學者,所以一個人製作是非常困難的,因此需要組員間的討論才能完成

40623226:每個人都分配到特定的工作,不像之前一樣一個人辛苦地慢慢做,有了團隊分工效率變得更快

40623225:本來從一開始甚麼東西都是自己來,到現在變成一個團隊憶起分工合作,一起將一個計算機做出來別新奇

說明各學員任務與執行過程

40623236:數字鍵 等號 計算 清除 等號跟計算最麻煩
因為會影響到很多人當有問題時我都要去幫忙

40623235:根號 平方
以及分數是相較於其他部分來說是算簡單的,所以常去幫其他組員要做的部分,一起解決問題,或是集結大家來討論

40623234:加減的運算
雖然加減運算的先乘除後加減的程式語言很難,但是問問同學後就會瞭解許多。

40623227:+-號 報告
因為報告要集結大家意見,所以在寫的過程中幫忙我的其實很多

40623226:小數點、
變號鍵、回復鍵、清除鍵按下後的處理方法,最後做的雖然比較複雜但因為最後推不會發生衝突

40623225:任務分配是乘跟除,需要去了解各個字所代表的意思,覺得這個部分特別的難

\hypertarget{ux7d50ux8ad6}{%
\chapter{結論}\label{ux7d50ux8ad6}}

期末報告結論

\hypertarget{ux7d50ux8ad6ux8207ux5efaux8b70}{%
\section{結論與建議}\label{ux7d50ux8ad6ux8207ux5efaux8b70}}

要寫一個計算機比想像中的困難,程式的邏輯以及運用方法,都是非常的困難,但在製作過程中,有大家的參與,組員間的討論溝通,解決遇到的問題,這都讓在寫計算機的時候變得簡單許多,能夠完成自己組別的一個計算機。
建議:需要隨時注意完成度

\hypertarget{ux53c3ux8003ux6587ux737b}{%
\chapter{參考文獻}\label{ux53c3ux8003ux6587ux737b}}


\end{document}
